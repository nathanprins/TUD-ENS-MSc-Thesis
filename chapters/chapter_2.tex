\chapter{Background}
\label{chp:chapter_2}

In this chapter the necessary background information will be covered to be able to fully understand the architecture of the proposed solutions, as well as defining useful terms. 

\section{Wireless Connection}
\label{sec:ch2_wireless_connection}
A wireless connection is setup between at least two devices. Within the bluetooth low-energy specification there exists a clear hierarchy, where one device is dominent and is called the Central. The central usually takes the shape of a phone or a laptop. The other side of the connection is with one or more Peripherals. These devices include headphones, smartwatches, IoT sensors, pushbuttons and much more. 

For the sake of this thesis I define three phases for a BLE connection. These phases are:
\begin{itemize}
    \item \textit{Unconnected}: before any connection is established between the Central and the Peripheral
    \item \textit{Connection Setup}: starts from the connection request and ends when the last non-application packet is sent. 
    \item \textit{Application}: starts when connection setup is finished and only application packets are sent. Application packets are packets that are necessary to fullfil the application. 
\end{itemize}
These terms are not officially defined by the BLE specification, but are used throughout this thesis. 

\section{Advertisers and Scanners}
When in an unconnected state, the Peripheral and Central can take on the roles of \textit{Advertiser} and \textit{Scanner} respectively. As an Advertiser, a Peripheral sends out advertising packets. Advertising packets can contain up to 31 bytes of advertising data. The structure of the advertising data is shown in Figure \ref{fig:advdata_layout}. The \texttt{AD Type} field is an 8 bit identifier which refers to one of the many predefined advertisement datatypes. This is followed by the length of the data for this type and then the actual data. 

\begin{figure}[]
    \centering
    \includegraphics[width=0.8\textwidth,height=6cm,keepaspectratio=true]{images/advertising_data}
    \caption{
        The layout of the AdvData field for advertisement packets. Advertisement packets allow for 31 bytes of consecutive AD Structures. These structures contain the length of the structure, followed by the AD Type, and finally the AD Data as defined in the specification for that AD Type \cite{bluetooth_spec}.
    }
    \label{fig:advdata_layout}
\end{figure}

There are multiple different advertising modes, but the only one used in this thesis is the \texttt{ADV\_IND}. This mode has two traits which are \textbf{Connectable} and \textbf{Scannable}. Connectable means that the Peripheral is open for connection requests. Scannable means that a Central can, as a response to an advertising packet, send a \textit{Scan Request} to request more data from the Perihperal. This data is then sent in the form of a \textit{Scan Response} and allows for another 31 bytes of advertising data.

\section{Connection Parameters}
The connection setup is initiated by the Central by responding with a \texttt{CONNECT\_IND} packet to an advertisement packet. From this moment a connection between the devices is made and the setup can begin. The parameters which decide the connection configuration are contained within the \texttt{CONNECT\_IND} packet (\texttt{LLData} field). The most important configuration values to understand are \textit{Connection Interval} (CI) and \textit{Peripheral Latency} (PL). 

\begin{figure}[]
    \centering
    \includegraphics[width=0.8\textwidth,height=6cm,keepaspectratio=true]{images/connection_interval_slave_latency}
    \caption{
        An example timeline of communications between a \color{red} Peripheral \color{black} and \color{blue} Central \color{black}. At every connection interval a new chain of transmission events can start. This chain can theoretically continue up until the end of the connection interval. If the Perpipheral Latency is set to $N$, it allows the Peripheral to skip responding to $N$ connection interval events from the Central in order to conserve energy. This can be seen in the third spike \cite{nordic_2022}.
    }
    \label{fig:ci_and_pl}
\end{figure}

As you can see in Figure \ref{fig:ci_and_pl}, the Connection Interval is the time between the start of chains of transmit (TX) and receive (RX) events. Multiple TX/RX events can happen within one chain, but when there is no more data to be sent then the current chain is over and new data can be transmitted at the following interval. Peripheral Latency defines how many Connection Intervals the Peripheral is allowed to skip. For example, take a CI of 2 seconds and a PL of 0, this means that every two seconds the Central transmits a packet and the Peripheral replies. Now take a CI of 2 and a PL of 1, in this case the Central transmits every two seconds, but the Peripheral is allowed to sleep every other packet, effectively making the time inbetween transmissions four seconds. 


\begin{table}
    \begin{center}
    \begin{tabular}{|l|l|}
        \hline
        \textbf{Parameter} & \textbf{Constraints} \\
        \hline
        Connection Interval & Range: 0x0006 0x0C80 \\
                            & $Time = N * 1.25ms$ \\
                            & Time Range: 7.5ms to 4000ms \\
        \hline
        Peripheral Latency  & Range: 0x0000 to 0x01F3 \\
        \hline
        Supervision Timeout & Range: 0x000A to 0x0C80 \\
                            & $Time = N * 10ms$ \\
                            & Time Range: 100ms to 32s \\
                            & $Timeout \leq (Latency + 1) * Interval$ \\
        \hline
    \end{tabular}
    \end{center}
    \caption{Connection parameters and parameter constraints.}
    \label{tbl:conn_params}
\end{table}

The units and ranges for the three connection parameters are defined in Table \ref{tbl:conn_params}.

\section{Services and Characteristics}
\begin{figure}[]
    \centering
    \includegraphics[width=0.5\textwidth,height=6cm,keepaspectratio=true]{images/gatt_service}
    \caption{
        The architecture of the GATT Server. Functionality is grouped as services. Services contain data as characteristics that can be read from or written to. Descriptors are used as metadata that describe each characteristics and for configuring the server to notify the client of updates from the characteristic they're grouped under \cite{townsend_cufi}.
    }
    \label{fig:gatt_server}
\end{figure}
The \textit{Generic Attribute Profile} (GATT) defines how profile and user data is exchanged over a BLE connection. GATT defines the GATT Server, within which functionality is split up into \textit{Services}. Services contain datapoints called \textit{Characteristics}. These characteristics can be interpreted and configured using fields called \textit{Descriptors}. See Figure \ref{fig:gatt_server} for a schematic layout of the GATT Server.

\begin{figure}[]
    \centering
    \includegraphics[width=0.5\textwidth,height=6cm,keepaspectratio=true]{images/heartrate_service}
    \caption{
        An example of a service. Each element is searchable through its UUID and then uniquely identifiable using its consecutively numbered handle \cite{townsend_cufi}.
    }
    \label{fig:hrs_layout}
\end{figure}
To give a real world example, take the Heart Rate Service in Figure \ref{fig:hrs_layout}. This service allows a Client to read the heartrate sensor of a device. The heartrate service contains a characteristic called Heartrate. One of the descriptors tells us that the unit is defined as \textbf{beats per minute}. We could read the characteristic value manually, but if we would like to be notified when a new measurement is done then we can set the Notify bit of the \textit{Client Characteristic Configuration} or CCC descriptor.

To be able to read or write to a Characteristic we need its handle. The handle is a number that is unique for a characteristic within a GATT Server. To find this handle we can perform a \textit{Find By Type Request} using its 16-bit Universally Unique Identifier (UUID). Bluetooth SIG has predefined UUIDs for many predefined Services. Each predefined Service has a specification which defines the shape of the service. This includes all the Characteristics, Descriptors and their respective UUIDs.

The process of finding all these handles is what is called \textit{Service Discovery}. The result of this process is a list of numbers that can be used to read and write to the server. After Service Discovery is done the GATT Servers need to be configured. This usually means writing to the CCC descriptor to setup notifications for the desired Characteristics. This process will henceforth be refered to as Configuration. 

Both the Central and the Peripheral can be Servers and Clients at the same time. For example; a phone can provide the central time for a smartwatch to display on the watchface, while the smartwatch measures the heartrate for the phone to display within a health application. This means that a Service Discovery and Configuration needs to be performed by both the Central and the Peripheral.

\section{Link Layer}
The Link Layer is the part of the Bluetooth stack which is tasked with managing the wireless link and actually sending data frames. The Link Layer sits inbetween the higher level protocols like GAP and GATT and the Physical Layer which actually controls the radio.
Procedures are defined within the Link Layer to make sure both the transmitting radio and receiving are configured correctly.

The \textit{Features Exchange} procedure is used to exchange which features are supported by both radios. These feature sets can vary widely between versions of Bluetooth and can have a large impact on performance. To guarantee that both sides of the link communicate in the same manner, these features are exchanged by the Link Layer at the start of each connection.

The size of the packets that can be received by a radio can also vary and is usually dependent on the available memory. To make sure the best throughput is achieved this needs to be communicated using the \textit{Data Length Exchange}. This exchanges the largest Air packet that each radio is able to receive at once.

\section{Connection Setup}
As previously defined in Section \ref{sec:ch2_wireless_connection}, the connection setup starts from the Connection Request and ends when the last non-application packet is sent. The packets that occur during Connection Setup can be divided into four groups which correspond to the subjects discussed in the previous sections.
\begin{itemize}
    \item Link Layer
    \item Service Discovery
    \item Configuration
    \item Application
\end{itemize} 
The order in which they occur are usually Service Discover, Configuration and then Application, with Link Layer communication starting parallel to the Service Discovery from the start.

\section{Operating Systems}
The FreeBie project was originally developed using the open-source BLE stack Packetcraft. Due to its architecture it was especially suitable to be modified in the way required for intermittent operation. This is because all OS tasks are managed using a single scheduler which uses a generic sleep method. This sleep method was altered to configure the real-time clock (RTC), dump the memory to FRAM, and to fully power down the SoC.

However, since FreeBie was originally developed, Packetcraft has gone into closed-source development. For this reason, a newer RTOS was chosen for the Central. Zephyr OS is a modern RTOS developed by the Linux Foundation and is now the officially supported stack for Nordic Semiconductors.

**something extra about modifying link layer stack